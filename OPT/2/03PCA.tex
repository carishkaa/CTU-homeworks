\documentclass[11pt]{article}
\usepackage{amssymb}
\usepackage{amsmath}
\usepackage{xcolor}
\usepackage[left=3.6cm,right=3.9cm,top=2.5cm,bottom=2cm,bindingoffset=0cm]{geometry}
\begin{document}

\begin{flushright}
B0B33OPT\\
Karina Balagazova
\end{flushright}
\part*{PCA: Motion Capture}

\section{Proložení bodů podprostorem}

\paragraph*{1.1}
Úlohu budeme řešit tak, že nejprve najdeme bázi hledaného podprostoru U a pak pomoci ní nalézneme body $ b_{1},...,b_{n} $.

\subparagraph*{a)}
Součet čtverců vzdálenosti bodů $ a_{1},...,a_{n} $ k podprostoru U je stejná hodnota jako součet čtverců délek projekce těch bodů na ortogonalní doplněk:
$$\| X^{T}a_{1} \| + ... + \| X^{T}a_{n} \| = \| X^{T}A \| = tr(X^{T}AA^{T}X)$$
Pomoci spektrálního rozkladu $VDV^{T} = AA^{T}$ dostaneme matici $V \in \mathbb{R}^{m\times m}$, jejíž $k$ posledních sloupců budou tvořit bázi našeho hledaného podprostoru.

\subparagraph*{b)}
Body $b_{1},...,b_{n}$ jsou ortogonálními projekcemi bodů $a_{1},...a_{n}$ do podprostoru U, tj. $b_{i} = UU^{T}a_{i}$, maticově $B = UU^{T}A$.

\subparagraph*{c)}
Víme, že $UU^{T}=1$. Každou stranu rovnice $B = UC$ zleva vynasobíme maticí $U^{T}$ a tak dostaneme body $c_{1},...,c_{n}$
$$C = U^{T}B = U^{T}UU^{T}A = U^{T}A$$ 


\paragraph*{1.2}
V případě že hledáme afinní podprostor, nejdřív musíme posunout body $ a_{1},...,a_{n} $ tak, aby jejich těžiště leželo v počatku, a dále už postupujeme stejně jako při hledání lin.posprostoru.

\section{Proložení bodů přímkou}

Zavoláme pro dané body funkci \textit{[U,C,b0]=fitaff(A,1)} a dostaváme:
$$ U = 
\begin{bmatrix}
-0.9617\\
-0.2740
\end{bmatrix}, 
b_{0} = 
\begin{bmatrix}
5.1231\\
4.9876
\end{bmatrix}
$$
Body $b_{1},...,b_{n}$ najdeme jako $b_{i} = Uc_{i} + b_{0}$, tj. rovnici přímky zjistíme takto: 
$$\begin{cases} x = u_{1}c + b_{0_{x}} \\ y = u_{2}c + b_{0_{y}}\end{cases} \Rightarrow y = \dfrac{u_{2}}{u_{1}} (x - b_{0_{x}}) + b_{0_{y}} = 0.2849x + 3.5280$$

\subsection{Nalezení přímky ve tvaru $\{b_{0}+uc$ $|$ $c \in \mathbb{R}\}$}
Najdeme přímku ve tvaru $\{b_{0}+uc$ $|$ $c \in \mathbb{R}\}$, kde norma $\|b_{0}\|$ je nejmenší možná (tj. aby $\|b_{0}\|$ byla vzdálenost přímky od počátku 0). Jinými slovy hledáme projekci počátku $(0,0)$ na přímku $y = 0.2849x + 3.5280$ a to bude bod $$ b_{0} = 
\begin{bmatrix}
1.0939\\
3.8397
\end{bmatrix}$$
Zkusíme do funkci \textit{[U,C,b0]=fitaff(A,1)} přidat nový vektor $b_{0}$. Dostaváme stejný vektor $U$ jako jsme měli předtím, tj. $$ u = 
\begin{bmatrix}
-0.9617\\
-0.2740
\end{bmatrix}$$

\subsection{Nalezení přímky ve tvaru $\{b \in \mathbb{R}^{2}$ $|$ $x^{T}b = y\}$}
Najdeme přímku ve tvaru $\{b \in \mathbb{R}^{2}$ $|$ $x^{T}b = y\}$. Za $x$ vezmeme vektor kolmý na vektor U, tj.
$$ x = 
\begin{bmatrix}
0.2740\\
-0.9617
\end{bmatrix},
$$ za $b$ dosadíme $b_{0}$ a tak najdeme 
$$y = 0.2740*5.1231-0.9617*4.9876 = -3.3928$$


\section{Komprese a analýza sekvence z motion capture}

\subsection{}
Optimální hodnotu úloh najdeme jako $$\lambda_{1}+...+\lambda_{m-k},$$ kde $\lambda_{1}\leqslant...\leqslant\lambda_{m}$ jsou vlastní čísla matice $AA^{T}.$
Při singularním rozkladu matice A její nenulová singulární čísla jsou druhé odmocniny nenulových vlastních čísel matice $AA^{T}$

\subsection{}
Minimální dimenze afinního podprostoru je $k=1$. Protože pokud body nemění svoje vzájemné umístění a pohybujou se jenom po přímce, tak můžeme ten pohyb reprezentovat jenom jako posunutí jednoho bodu na přímce.

\end{document}