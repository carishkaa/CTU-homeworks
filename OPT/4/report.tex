\documentclass[11pt]{article}
\usepackage{amssymb}
\usepackage{amsmath}
\usepackage{physics}
\usepackage{xcolor}
\usepackage[left=3.6cm,right=3.9cm,top=2.5cm,bottom=2cm,bindingoffset=0cm]{geometry}
\usepackage{graphicx}
\begin{document}

\begin{flushright}
B0B33OPT\\
Karina Balagazova
\end{flushright}
\part*{Minimaxní lineární regrese}

\section{Zadání}
Dáno je $m$ bodů $(x_{1}, y_{1}),...,(x_{m}, y_{m})$, kde $x_{i} \in \mathbb{R}^{n}$. Úkolem je nalézt takovou afinní funkci
$$f(x) = a^{T}x + b,$$
parametrizovanou vektorem $a \in \mathbb{R}^{n}$ a skalárem $b \in \mathbb{R}$, která 'nejlépe' aproximuje funkční závislost odpovídající daným bodům.
Jako míru kvality aproximace zvolíme maximální absolutní odchylku
$$r = \max_{i = 1}^{m} |f(x_{i}) - y_{i}|$$
Cílem je nalézt parametry $a$, $b$ afinní funkce, pro které je číslo $r$ nejmenší.

\section{Převedení na LP}

Úlohu 
$$\min_{x \in \mathbb{R}^{n}} \max_{i = 1}^{m} |f(x_{i}) - y_{i}| = 
\min_{x \in \mathbb{R}^{n}} \max_{i = 1}^{m} |a_{i}x_{i} + b_{i} - y_{i}|$$
převedeme na lineární program pomocí zavedení proměnné $z \in \mathbb{R}$ takové, aby platila nerovnost $f(x) \leq z$. Nová úloha bude vypadat takhle:
$$
\begin{array}{lcl} 
\min z \\ 
\textrm{za podm.} -z1 \leq a^{T}x + b - y \leq z1
\end{array}
$$
Po upravení podmínek dostaneme
$$
\begin{cases}
 a^{T}x + b - y \leq z \\ 
 a^{T}x + b - y \geq -z
\end{cases}
=
\begin{cases}
 a^{T}x + b - z \leq y \\ 
 -a^{T}x - b - z \leq -y
\end{cases}
$$
což lze napsat jako
$$
	\begin{bmatrix} x & 1 & -1 \\ -x & -1 & -1 \end{bmatrix}
	\begin{bmatrix} a\\ b\\ z \end{bmatrix}
	\leq
	\begin{bmatrix} y\\ -y \end{bmatrix}
$$

\section{Souvislost bodů x s okraji pásu}
Pro obecná $n$ \emph{Chebyshev Equioscillation Theorem} říká, že pokud funkce $f$ je nejlepší možná aproximace bodů $x$, pak vždycky existuje množina z alespoň $n+2$ bodů, pro které funkce $r(x)$ nabývá absolutního maxima (se střídávým znamínkem).\\
Pro $n=1$ takových bodů musí být nejméně tři a to je přesně to, co pozorujeme při vykreslování. Samozřejmě takových bodů může být i víc, například pří aproximaci periodických funkcí (sin, cos, ...) nebo kdybychom chtěli najít optimální řešení pro body, ležící na dvou paralelních přímkach (okraje pásu by prochazely přes všechny dané body).\\
Taky si pro mě bylo důležité všimnout, že přídáním nebo odebráním libovolných bodů uvnitř pásu se nic nemění, protože infinity norma nepracuje se střednimí hodnotami a výsledek ovlivňujou právě ty nejvíc vychýlené body.


\end{document}